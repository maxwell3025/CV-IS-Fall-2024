
\section{Background on State Space Models}
In this paper, we use a network architecture called Mamba, which is centered
around a type of sequence-to-sequence model called a state space model(SSM).

\subsection{State Space Models}
State space models(abbreviated SSMs) are a type of sequence-to-sequence model
that have recently found promising applications in language
modeling\cite{mamba}, image processing\cite{medmamba}, and more\cite{s4}.

The recent wave of SSMs has been discrete SSMs, which, as the name suggests, are
discrete versions of continuous SSMs.
Continuous SSMs introduced by Kalman\cite{kalman} for signal processing, and
they assume that the system that they are modeling has some internal state $s$
that is affected by an input signal $x$, and produces an output signal $y$ based
on $x$ and $s$.
The governing equations are as follows:
$$\begin{aligned}
    \frac{ds}{dt} &= \Ab\vec{s} + \Bb x \\
    y &= \Cb\vec{s} + \Db x
\end{aligned}$$
where $x, y, s$ are vectors, and $\Ab, \Bb, \Cb, \Db$ are matrices.
These models can simulate a variety of systems, such as physics equations,
financial equations, and more.

A key observation is that all of the relationships are linear. This is one of
the key properties used by discrete SSMs.

\subsubsection{Discretization}
Since we can't store general continuous sequences in hardware, real
implementations of SSMs typically approximate the dynamics across discrete
timesteps.
A common method for discretizing SSMs is used by Gu et al. \cite{s4}
$$\begin{aligned}
    s_k &= \overline \Ab s_{k-1} + \overline \Bb x_k
    &
    \overline \Ab &=
    \left(I - \frac{\Delta}{2}\Ab\right)^{-1}
    \left(I + \frac{\Delta}{2}\Ab\right)
    \\
    y_k &= \overline \Cb s_k
    &
    \overline \Bb &=
    \left(I - \frac{\Delta}{2}\Ab\right)^{-1}
    \Delta \Bb
    &
    \overline \Cb &= \Cb
\end{aligned}$$
This scheme assumes that the input signal is constant within each timestep and
uses the trapezoidal rule for integration.
Note that $\Db$ is omitted. This is since the "pass-through" function of $\Db$
can be easily emulated with this discretization.

\subsection{Hyperparameters}
In the style of Gu et al. \cite{s4}, we will use the following notation for
dimensions:
\begin{itemize}
    \item $B$ - The batch size.
    \item $L$ - The length of the sequence.
    \item $D$ - The number of individual SSMs to run in parallel. This is the
    actual number of input channels.
    \item $N$ - The state size for each SSM.
\end{itemize}
Note that the input and output sizes are omitted.
These are generally both set to 1.

\subsection{S4}
S4\cite{s4} is one implementation of state space models that implements the
scheme detailed above.
S4 uses an initialization for $\Ab$ based on the HiPPO framework, which is
designed to efficiently represent long-range dependencies.
The key development introduced by S4 is that the S4 parametrizes $\Ab$ as a
normal-plus-low-rank(NPLR) matrix. The authors then use the spectral properties
of these matrices in an algorithm that brings the training complexity from
$O(BLN^2)$ down to $O(BN(N \log N + L \log L) + B(L \log L)N)$\cite{s4}.
This allows the authors to achieve much higher state sizes than normal.
% TODO: How Much?
As for parameterization, S4 makes $\Lambda$, $\Pb$(Components of
$\Ab = \Lambda - \Pb\Pb^*$), $\Bb$ and $\Cb$ trainable parameters.
$\Delta$ is indirectly trainable by scaling $\Bb$ and $\Ab$.

\subsection{Mamba}
Mamba\cite{mamba} is the model architecture that we use in our experiments.
Mamba introduces a new SSM architecture, called S6, and it provides a new block
design for introducing gating.

The S6 architecture is an improvement on the S4 architecture that introduces
timesteps that vary within sequences.
This breaks the algorithm introduced in S4, so the authors fall back on more
general, existing optimizations in order to maintain performance with large
state sizes.
Variable timesteps allow the model to selectively "forget" and "ignore"
different values.
If the timestep for a given input is set to a high value, the existing state
will decay to close to 0, allowing the model to "forget" all previous information.
If the timestep for a given input is set to a low value, the state has very little
time to be influences, and in the case of the timestep being 0 or almost 0, it
is as if the model "skips over" the given input token.

The block model introduced by the authors is an improvement on the existing
H3\cite{h3} architecture, and adds explicit selection functionality.

This allows the model to perform well on copying tasks, which we hypothesize
is crucial for in-context character recognition.
